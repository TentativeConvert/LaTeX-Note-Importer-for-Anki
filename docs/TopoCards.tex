% -*- coding: utf-8-unix -*-
%%%%%%%%%%%%%%%%%%%%%%%%%%%%%%%%%%%%%%%%%%%%%%%%%%%%
% The first part of the header needs to be copied
%       into the note options in Anki.
%%%%%%%%%%%%%%%%%%%%%%%%%%%%%%%%%%%%%%%%%%%%%%%%%%%%

% layout in Anki:
\documentclass[11pt]{article}
\usepackage[a4paper]{geometry}
\geometry{paperwidth=.5\paperwidth,paperheight=100in,left=2em,right=2em,bottom=1em,top=2em}
\pagestyle{empty}
\setlength{\parindent}{0in}
 
% hyphenation:
\usepackage[ngerman]{babel}

% encoding:
\usepackage[T1]{fontenc}
\usepackage[utf8]{inputenc}
\usepackage{lmodern}

% packages:
\usepackage{amsmath,amsfonts}  
  
%%%%%%%%%%%%%%%%%%%%%%%%%%%%%%%%%%%%%%%%%%%%%%%%%%%%
% Following part of header NOT to be copied into
%            the note options in Anki.
%          ! Anki will throw an errow !
%%%%%%%%%%%%%%%%%%%%%%%%%%%%%%%%%%%%%%%%%%%%%%%%%%%%%
%
%  pdf layout:
%
\geometry{paperheight=74.25mm}
\usepackage{pgfpages}
\pagestyle{empty}
\pgfpagesuselayout{8 on 1}[a4paper,border shrink=0cm]
\makeatletter
\@tempcnta=1\relax
\loop\ifnum\@tempcnta<9\relax
\pgf@pset{\the\@tempcnta}{bordercode}{\pgfusepath{stroke}}
\advance\@tempcnta by 1\relax
\repeat
\makeatother
% 
%  notes, fields, tags:
%
\newcommand{\xfield}[1]{
        #1\par
        \vfill
        {\tiny\texttt{\parbox[t]{\textwidth}{\localtag\\\globaltag\hfill\uuid}}}
        \newpage}
\newenvironment{field}{}{\newpage}
\newif\ifnote
\newenvironment{note}{\notetrue}{\notefalse}
\newcommand{\localtag}{}
\newcommand{\globaltag}{}
\newcommand{\uuid}{}
\newcommand{\tags}[1]{
    \ifnote 
        \renewcommand{\localtag}{#1}
    \else
        \renewcommand{\globaltag}{#1}
    \fi 
    }
\newcommand{\xplain}[1]{\renewcommand{\uuid}{#1}}
%
%%%%%%%%%%%%%%%%%%%%%%%%%%%%%%%%%%%%%%%%%%%%%%%%%%%%
% The following line again needs to be copied 
% into Anki:
\begin{document}
%%%%%%%%%%%%%%%%%%%%%%%%%%%%%%%%%%%%%%%%%%%%%%%%%%%%

\tags{Topo-0-1-Grundbegriffe}

\begin{note}
  \xplain{214a670c-1416-4e53-9f8a-7e598f437f40}
  \tags{Definition}
  \xfield{Topologie auf einer Menge \(X\)}
  \begin{field}
    Eine Menge von Teilmengen von \(X\), die \(\emptyset\) und ganz \(X\) enthält, sowie das Vereinigungs- und das Schnittaxiom erfüllt.
  \end{field}
  \xfield{Topologie:  Vereinigungsaxiom}
  \begin{field}
    Beliebige Vereinigungen offener Mengen sind offen.
  \end{field}
  \xfield{Topologie:  Schnittaxiom}
  \begin{field}  
    \emph{Endliche} Schnitte offener Mengen sind offen.
  \end{field}
\end{note}

\begin{note}
  \xplain{f13fb51e-4905-4a24-89dc-13d334e8cc92}
  \tags{Definition}
  \xfield{Abschluss einer Menge}
  \begin{field}
    Schnitt aller abgeschlossenen Übermengen.
  \end{field}
  \xfield{Inneres einer Menge}
  \begin{field}
    Vereinigung aller offener Teilmengen.
  \end{field}
  \xfield{Rand einer Menge}
  \begin{field}
    Abschluss ohne Inneres.
  \end{field}
\end{note}

\begin{note}
  \xplain{ddb58e94-e730-4124-adc3-c838bab8996f}
  \tags{Beispiel}
  \xfield{Rand der abgeschlossenen Scheibe in \(\mathbb R^2\)}
  \begin{field}
    Der Kreis.
  \end{field}
  \xfield{Rand der offenen Scheibe in \(\mathbb R^2\)}
  \begin{field}
    Der Kreis.
  \end{field}
  \xfield{Rand der offenen Scheibe in sich selbst.}
  \begin{field}
    \(\emptyset\)
  \end{field}
\end{note}

\tags{Topo-0-2-Zusammenhang}
\begin{note}
  \xplain{b1aca801-04c7-476f-8a11-15baef7741ee}
  \tags{Beispiel}
  \xfield{Eine Raum ist zusammenhängend, falls \dots}
  \begin{field}
    \dots er sich nicht als \emph{disjunkte} Vereinigung zweier offener, nicht-leerer, echter Teilmengen schreiben lässt.
  \end{field}
\end{note}

\begin{note}
  \xplain{5a7de3ec-5e7b-4d3d-8aaa-a3ff469956fd}
  \tags{Definition}
  \xfield{Eine Raum ist weg-zusammenhängend, falls \dots}
  \begin{field}
    \dots sich je zwei Punkte durch einen stetigen Weg verbinden lassen.
  \end{field}
\end{note}

\end{document}
